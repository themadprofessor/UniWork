\documentclass[11pt]{article}
\usepackage[basic]{wordlike}

\begin{document}

\title{Book Review - Release It! Second Edition - Michael T. Nygard}

\author{2258082 - Stuart Reilly}

\date{\today}

\maketitle

\section{Overview}
Release It! is a guide on how a programmer can design a system which is designed for production 
from the outset of development, while providing case studies to show real-world examples of why
this is important.
The book is broken into four part; Create Stability, Design for Production, Deliver Your 
System, Solve Systemic Problems.
Each of these part are further broken down in multiple chapters, starting with a case study,
followed by multiple chapters, focusing on the solutions to the issues found in the case study.
Breaking each part into multiple chapters in this structure allows the chapter to have a flow
from showing there is an issue and its implications, to possible solutions and how to apply the
solutions to the reader's own systems.
The parts also provide an overarching flow of how to handle the failure of a system and 
minimising the impact of the failure, reducing the likelihood of a failure in a system,
deploying a system in a manner to reduce the likelihood of failure, and how to handle a failure
in production.
By the end of the book, the reader should have an understanding of why it is important and how to 
design a system for production early in the system's development, and have a systematic method to 
resolve a failure in production.

\section{Analysis}
Overall, Release It! is a balanced book.
Although the majority of the book is focused on the design of a system, rather the deployment of 
the system, this is justified by the fact the system must be designed for production for the 
deployment techniques to be effective.
The security chapter, while applicable to the topics the book works with, is somewhat out of place
compared to the other chapters.
It focuses on how to protect a system from data breaches, and protecting sensitive data if a data
breach was to occur.
This does align itself with the overall topic of designing systems for production, however the 
rest of the book focuses on design patterns to avoid catastrophic failure, handling small failures
gracefully and minimising downtime.
Switching to a chapter focused on the implementation details of a secure system is not match the 
more abstract focus of the book.
Furthermore, the security of a system is a broad topic which cannot be discussed in enough depth to
be relavent in a single chapter, and while the security concepts may not become out of date soon 
after publishing, the specific implementation details and decisions will.
With that said, the implementation details in the chapter are fairly modern.
For example, it states ``Stop using SHA-1. Just Stop. Its not longer adequate.'' 
\cite[p.~226]{release}.
SHA-1 had its first public collision on 23 February 2017, and has only been deemed inadequate since
then.
The design for deployment chapter is one of particular value as it explains why deployment is not
only for an operations team, but also the development team.
Dispelling the belief of planned downtime being an acceptable part of deployment and alternative
methods of deployment to avoid it, is the main aim of the chapter, which it succeeds in discussing
effectively.
The key concept from the chapter is planned downtime is as bad as unexpected downtime for an end
user, as such should be avoided.
In order to achieve this, the design of the system must take into account automated and continuous
deployment.
The chapter effectively breaks down the phases of deployment and how to account for them in the
design for a system and methods for rolling out a new release of a system seamlessly for the end
user.
Also, the chapter is expressed in an unopinionated manner, allowing the reader to decide which
technique is better for a particular system.
Both convergent and immutable infrastructure are discussed in the chapter, without the author
defining one as better than the other, instead convergent is better suited for smaller systems,
and immutable is better for larger systems.
Furthermore, the chapter discusses topics which have a large range of available tools, but the 
author minimises discussing particular tools.
By not discussing specific tools, the author is able to focus on the concepts the tools aim to 
solve, allowing the author to take the concepts on board and find tools which suit the needs of
their system, rather than the needs of the author's system.
Each art of the book begins with a case study of a system which didn't take into account the
topic of the chapter.
The case study shows how the system failed, exposing the problem discussed in the part, and how
the author diagnosed and solved the issue.
By starting with a case study, the reader is shown the problem exists in the real world, not an
abstract concept, and the difficulty the author had while diagnosing and solving the problem.
The remainer of the part of the book is dedicated to discussing possible methods of preventing the
problem in the design phase and often how to handle the problem, reducing the complexity of 
diagnosing the problem, at runtime.

\section{Summary}
Release It! is a book focused for experts, but should be read by any developer.
While the book assumes some technical understanding of systems, the topics in the book are vital
for all developers to understand in order to provide a better experience for both the end user and
themselves.
This is because, the system designed by a developer who took on board the topics in the book will
has a vastly more effective deployment than a system designed by a developer who hasn't taken the 
topics on board, reducing the downtime for the end user and frequency of both failed deployments
and catastrophic failures.
The un-opinionated nature of the book avoids guiding the reader into specific tools, but still 
provides enough detail to give the reader a starting point for evaluating different tools.
Overall, Release It! is a book which is a vital part of any developer's toolkit.

\bibliographystyle{plain}
\bibliography{bookreview}
\end{document}
