\documentclass[11pt, a4paper]{article}
\usepackage{mathtools}
\usepackage[a4paper,includeheadfoot,margin=2.54cm]{geometry}

\begin{document}

\section*{Question 1}
\subsection*{Part (a)}
\begin{align*}
	p \to (p \lor q) &\equiv \neg p \lor (p \lor q)               & \text{implication law} \\
			 &\equiv (\neg p \lor p) \lor (\neg p \lor q) & \text{distributive law} \\
			 &\equiv \texttt{true} \lor (\neg p \lor q)   & \text{tautology law} \\
			 &\equiv \texttt{true}                        & \text{identity law}
\end{align*}

\subsection*{Part (b)}
\begin{align*}
	((p \to q) \lor (\neg p \to r)) \to (q \lor r) &\equiv ((\neg p \lor q) \lor (\neg \neg p \to r)) \to (q \lor r)                           & \text{implication law} \\
	                                               &\equiv ((\neg p \lor q) \lor (p \to r)) \to (q \lor r)                                     & \text{double negation law} \\
						       &\equiv (\neg p \lor (p \lor r)) \lor (q \lor (p \lor r)) \to (q \lor r)                    & \text{distributive law} \\
						       &\equiv (\neg p \lor p) \lor (\neg p \lor r) \lor (q \lor p) \lor (q \lor r) \to (q \lor r) & \text{distributive law} \\
						       &\equiv \texttt{true} \to (q \lor r)                                                        & \text{identity law} \\
						       &\equiv \neg \texttt{true} \lor (q \lor r)                                                  & \text{implication law} \\
						       &\equiv \texttt{false} \lor (q \lor r)                                                      & \\
						       &\equiv q \lor r                                                                            & \text{identity law}
\end{align*}

\subsection*{Part (c)}
\begin{align*}
	\neg \forall x \in U.(P(x) \lor \neg Q(x)) &\equiv \exists x \in U.(\neg (P(x) \lor \neg Q(x)))     & \text{quantifier law} \\
	                                           &\equiv \exists x \in U.(\neg P(x) \land \neg \neg Q(x)) & \text{De Morgan's law} \\
						   &\equiv \exists x \in U.(\neg P(x) \land Q(x))           & \text{double negation law}
\end{align*}

\section*{Question 2}
\subsection*{Part (a)}
This formulae is true as every postive integer must be either even or odd, but not both.
Therefore, if any given positive integer is not odd, it must be even.

\subsection*{Part (b)}
This formulae is false, because every integer can have one subtracted from itself to produce a smaller integer.
Therefore, there cannot be an integer which is smaller than all other integers.

\subsection*{Part (c)}
This formulae is true, as 1 is not a prime number but is a natural number, therefore any prime natural number will be larger than it and not equal to it.

\subsection*{Part (d)}


\end{document}
