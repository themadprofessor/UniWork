%%
%% Author: stuart
%% 26/02/19
%%

% Preamble
\documentclass[10pt,a4paper]{article}

% Packages
\usepackage{amsmath}
\usepackage[top=1in, bottom=1.25in, left=1.25in, right=1.25in]{geometry}

% Document
\begin{document}

    \title{Network Systems (H) Laboratory Exercise 3\\Understanding the Topology of the Internet}
    \author{Stuart Reilly - 2258082R}
    \date{\today}
    \maketitle

    \section{IP Addresses}\label{sec:ip-addresses}
    Some of the domains have multiple IP addresses attached, such as google.com, mariadb.com and gov.cn.
    Having multiple IP addresses means the website is hosted on multiple data centres, allowing the site to be accessed
    by more devices at once.
    A DNS will likely either shuffle the multiple IP addresses or use a Round-Robin technique when query for a domain
    with multiple addresses.
    Running dnslookup from multiple locations provides different IP addresses, as each location had a different DNS
    server, and will be closer to different data centres.
    Out of the total 20 domains, only 8 had IPv6 addresses.
    This is likely because their is no incentive for a domain host to use IPv6 when IPv4 works and is more ubiquitous.

    \section{Router-level Topology Maps}\label{sec:router-level-topology-maps}
    The longest path in the IPv4 graph is 19 hops and the longest path is 17 hops.
    In the IPv6 graph, there all routes a disjoint and there is only one route to each address.
    In contrast, the IPv4 graph creprsentingontains cycles, which means their are multiple routes to some nodes.
    This shows the routes are congested and the packets have to routed down alternative route.

    \section{IPv4 and IPv6}\label{sec:ipv4-and-ipv6}
    The two graphs have similar structures within them where a domain has both IPv6 and IPv4 addresses.
    This is to be expected as they are likely to be representing the same routes.

    \section{Traceroute}\label{sec:traceroute}
    Traceroute is a networking tool for find the path between two devices on the Internet.
    This done by making use of the Internet Control Message Protocol, ICMP, and another protocol, often either TCP or UDP .
    A packet is sent with a incrementing time-to-live, TTL, value to the destination machine, then traceroute listens
    for a response.
    If the response is a time exceeded ICMP packet, the address of the sender is the next hop, so the address is printed
    and the TTL is incremented.
    If the response is not a time exceeded ICMP packet, the address of the sender is the destination device, so
    traceroute terminates.
    TTL is stored as an 8-bit field in the IP header of the packet, therefore traceroute has a an upper limit of 225
    hops.
    Both IPv4 and IPv6 use and 8-bit field, the difference is the location in the header.
    9th octet of 20 and 8th octet of 40 for IPv4 and IPv6 respectively.

    First, traceroute sends a packet to destination address with a TTL of 1.
    This packet will always fail to reach the destination unless the destination machine is directly connected to the
    source machine.
    The first device to recieve the packet (likely the router on the source machine's network) will decrement the TTL of
    the packet, which will be come 0, causing the router to send an ICMP packet back to the source machine, informing it
    that the TTL of the packet was exceeded.
    Traceroute will reattempt to send the packet a number of times, depending on the command line arguments given, before
    increasing the TTL of the packet.
    Then traceroute sends a packet to the destination device with a TTL of 2, therefore the first device will decrement
    the packet's TTL and send it on to the next device.
    The TTL of the packet sent is incremented until either the maximum number of hops is reached, or the returned ICMP
    packet is no long a time exceed response, at which time traceroute has found the destination machine and
    will terminate.
    Each time the TTL is incremented, the address of the device which sends the ICMP time exceeded packet is the address
    of the device at the current hop, and therefore if printed by traceroute.

    Traceroute can use any protocol to send packets, since TTL is handled by the IP header.
    For example on Unix-like systems, traceroute use UDP packets by default, but supports UDP, TCP, ICMP (using ICMP
    echo), UDP-Lite, DCCP and raw (doesn't send protocol specific headers, just IP headers).
    Windows tracert send ICMP Echo Requests which are also used by the ping command on most systems.
    By supporting multiple protocols, traceroute can target machines with more restrictive firewall configurations.
    For example, if a device's firewall does not allow UDP packets to be sent, another protocol can be used which is
    allowed.
    Traceroute does not change protocol automatically, it is up to the caller to determine a suitable protocol.

\end{document}