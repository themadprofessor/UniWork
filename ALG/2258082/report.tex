\documentclass{article}
\usepackage{a4wide,amsmath,amssymb,graphicx}

\title{Algorithmics I - Assessed Exercise\\ \vspace{4mm}
Status and Implementation Reports}

\author{\textbf{Stuart Reilly} \\ \textbf{2258082} }

\date{\today}

\begin{document}
    \maketitle

    \section*{Status report}

    In the event of a non-working program, this section should state clearly what happens when the program is compiled (in the case of compile-time errors) or run (in the case of run-time errors).

    Otherwise, this section should state whether you believe that your programs are working correctly. If so, indicate the basis for your belief, if not comment on what you think might be the problem.

    \section*{Implementation report}

    \subsection*{Dijkstra's Algorithm}

    Initially, the adjacency list of a Vertex was a HashMap<Integer, Node> as one assumed the algorithm would require
    frequent get and contains calls.
    Being a HashMap, these calls would be O(1) based on the hash function used for an Integer (in the case of Integer, it is
    just the int value of the Integer).
    However, after completing the algorithm's implementation, it was shown the adjacency list was used exclusively for
    iteration, which is more efficiently handled by an ArrayList.

    In order to keep track of the current path, a HashMap<Vertex, Vertex> is used to store which vertex is before a given
    vertex in the path.
    The use of a HashMap over a list allows for O(1) insertion and access, which are the only uses of this object.

    The Graph object contains an array of all the vertices in the graph as this allows for O(1) access (each Vertex keeps
    track of its index in this array).

    Once the Graph object is populated by parsing the input file, the starting vertex would be removed from the collection
    of unvisited vertices and marked as the current vertex.
    Then, the closest vertex to the current vertex would be then found.
    The distance to each of the nodes adjacent to the closest vertex from the starting node is then calculated.
    If the new distance is shorter than the previously known distance to the node, its distance is updated.
    The closest vertex is then removed from the collection of unvisited vertices.
    This is then repeated for the vertex with the shortest known distance which is yet to be visited, until the vertex to be
    operated on is the final vertex.


    \subsection*{Backtrack Search}

    Initially, the iteration with each recursion was implemented using the Stream API, in an attempt to provide more
    readable code for initial implementation.
    An alternative implementation using a foeach loop executed in a third of the time was then used.

    The set of unvisited vertices is implemented as a HashSet<Integer> since this would allow for O(1) insertion, removal
    and contains operations, which are called each iteration of the loop within each recursion.

    Once the Graph object is populated by parsing the input file, the starting vertex would be removed from the collection
    of unvisited vertices and marked as the current vertex.
    Then for each adjacent unvisited vertex to the current node where its addition to the current path produces a path of
    length shorter than the best known path, if the vertex is the end node, the best path is updated to be the current path
    including the vertex, otherwise recurse with the current path appended with the current vertex.

    \section*{Empirical results}

    \subsection*{Dijkstra}
    \subsubsection*{data1000.txt}
    \begin{verbatim}
        Shortest path between vertex 24 and vertex 152 is 17
        Shortest path: 24 922 810 957 963 371 837 152

        Elapsed time: 353 milliseconds
    \end{verbatim}

    \subsubsection*{data20.txt}
    \begin{verbatim}
        Shortest path between vertex 3 and vertex 4 is 1199
        Shortest path: 3 0 4

        Elapsed time: 70 milliseconds
    \end{verbatim}

    \subsubsection*{data40.txt}
    \begin{verbatim}
        Shortest path between vertex 3 and vertex 4 is 1157
        Shortest path: 3 36 4

        Elapsed time: 76 milliseconds
    \end{verbatim}

    \subsubsection*{data60.txt}
    \begin{verbatim}
        Shortest path between vertex 3 and vertex 4 is 1152
        Shortest path: 3 49 4

        Elapsed time: 78 milliseconds
    \end{verbatim}

    \subsubsection*{data6_1.txt}
    \begin{verbatim}
        Shortest path between vertex 2 and vertex 5 is 11
        Shortest path: 2 1 0 5

        Elapsed time: 69 milliseconds
    \end{verbatim}

    \subsubsection*{data6_2.txt}
    \begin{verbatim}
        No path found

        Elapsed time: 63 milliseconds
    \end{verbatim}

    \subsubsection*{data80.txt}
    \begin{verbatim}
        Shortest path between vertex 4 and vertex 3 is 1152
        Shortest path: 4 49 3

        Elapsed time: 84 milliseconds
    \end{verbatim}

    \subsubsection{Backtrack}
    \subsubsection*{data1000.txt}
    \begin{verbatim}
        Shortest path between vertex 24 and vertex 152 is 17
        Shortest path: 24 582 964 837 152

        Elapsed time: 284196 milliseconds
    \end{verbatim}

    \subsubsection*{data20.txt}
    \begin{verbatim}
        Shortest path between vertex 3 and vertex 4 is 1199
        Shortest path: 3 0 4

        Elapsed time: 22 milliseconds
    \end{verbatim}

    \subsubsection*{data40.txt}
    \begin{verbatim}
        Shortest path between vertex 3 and vertex 4 is 1157
        Shortest path: 3 36 4

        Elapsed time: 78 milliseconds
    \end{verbatim}

    \subsubsection*{data60.txt}
    \begin{verbatim}
        Shortest path between vertex 3 and vertex 4 is 1152
        Shortest path: 3 49 4

        Elapsed time: 342 milliseconds
    \end{verbatim}

    \subsubsection*{data6_1.txt}
    \begin{verbatim}
        Shortest path between vertex 2 and vertex 5 is 11
        Shortest path: 2 1 0 5

        Elapsed time: 18 milliseconds
    \end{verbatim}

    \subsubsection*{data6_2.txt}
    \begin{verbatim}
        Shortest path between vertex 2 and vertex 5 is 2147483647
        Shortest path:

        Elapsed time: 15 milliseconds
    \end{verbatim}

    \subsubsection*{data80.txt}
    \begin{verbatim}
        Shortest path between vertex 4 and vertex 3 is 1152
        Shortest path: 4 49 3

        Elapsed time: 1177 milliseconds
    \end{verbatim}
\end{document}
