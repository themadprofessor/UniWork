\documentclass{article}
\usepackage{a4wide,amsmath,amssymb,graphicx}

\title{Algorithmics I - Assessed Exercise\\ \vspace{4mm} 
Status and Implementation Reports}

\author{\bf Stuart Reilly\\ \bf 2258082}

\date{\today}

\begin{document}
\maketitle

\section*{Status report}

In the event of a non-working program, this section should state clearly what happens when the program is compiled (in the case of compile-time errors) or run (in the case of run-time errors).  

Otherwise, this section should state whether you believe that your programs are working correctly. If so, indicate the basis for your belief, if not comment on what you think might be the problem.

\section*{Implementation report}

\subsection*{Dijkstra's Algorithm}

Initially, the adjacency list of a Vertex was a HashMap<Integer, Node> as one assumed the algorithm would require frequent get and contains calls.
Being a HashMap, these calls would be O(1) based on the hash function used for an Integer (in the case of Integer, it is just the int value of the Integer).
However, after completing the algorithm's implementation, it was shown the adjacency list was used exclusively for iteration, which is more efficently handled by an ArrayList.

In order to keep track of the current path, a HashMap<Vertex, Vertex> is used to store which vertex is before a given vertex in the path.
The use of a HashMap over a list allows for O(1) insertion and access, which are the only uses of this object.

The Graph object contains an array of all the vertices in the graph as this allows for O(1) access (each Vertex keeps track of its index in this array).

Once the Graph object is populated by parsing the input file, the starting vertex would be removed from the collection of unvisited vertices and marked as the current vertex.
Then, the closest vertex to the current vertex would be then found.
The distance from the starting vertex to the vertex closest to current would then be stored if it is shorter than the previous path to this vertex.


Here, explain how you implemented the Dijkstra's algorithm to compute and return the required information. Include a discussion of any steps that you took to improve efficiency including the reasoning behind your choice of data structures.

\subsection*{Backtrack Search}

Here, explain how you implemented the Backtrack search to compute and return the required information. Include a discussion of any steps that you took to improve efficiency including the reasoning behind your choice of data structures.

\section*{Empirical results}

This section is part of the marking scheme "Outputs from test data: 2 marks".

Give the running times of your two programs on the data sets provided. If the program fails to terminate in, say, two minutes, then simply report non-termination. 

\end{document}
