\documentclass[12pt]{article}
\title{Facebook Mining Debate}
\author{Stuart Reilly}
\date{\today}

\begin{document}
\maketitle
\section{Responsibilities of software engineers}
	A software engineer is a person who creates the mechanism to collect, store, recall and display the data collected through Facebook's user interface.
	Due to their role being intrinsic to the collection and handling of user's personal data, they have a responsibility to ensure the data can only be viewed and modified by authorised users.
	These authorised users will be the user who's personal data is being collected, Facebook's data analysis team and anyone with enough authority to view any data they request.
	The user who's data is being collected, must be able to access and modify their data as its is data which is about them.
	If the user wishes to modify the data, or view the data at anytime, they must be able to.
	The software developers would have to implement a robust and secure method of accessing and modifying the data.
	With an insecure method, the data could be intercepted and captured or modified without the user's permission.
	With a non-robust method, the method could fail during the transaction, which can lead to data corruption or unexpected modification.
	Within the terms and conditions, which all users of Facebook must agree too, there is a clause to allow Facebook to use any and all data which any and all user supply for data analysis.
	As a software engineer, there must be a method of anonymising all user data which is requested by Facebook's data analysis team.
	All data used in data analysis must be anonymous in order to protect the identities of the users being analysed.
	With this said, people with specific authority will request specific user data which is not anonymised.
	Sadly, there is no action a software developer can take to circumvent this action and protect user's identities.
	In order to comply with data protection laws, the data collected by Facebook must be stored in a secure manner to ensure data is not accessed by unauthorised users.
	Software developers have the responsibility to ensure all software packages remain up to date in order to ensure all known security vulnerabilities are resolved.
	They also have to ensure they remain up to date with the latest security ''best practices'' and provide a possible implementation for these practices.
\end{document}
