\documentclass[a4paper, 11pt]{article}
\usepackage[style = authoryear]{biblatex}
\usepackage[a4paper,includeheadfoot,margin = 2.54cm]{geometry}
\addbibresource{2258082r.bib}

\begin{document}
\title{IT professional issues raise by Twitter's unintentional incorrect use of personal information}
\date{\today}
\maketitle

\section{Introduction}
On the 8th of October 2019, Twitter delivered a public apology for unintentionally using user's personally information incorrectly \parencite{support_twitter_2019}.
\textcite{support_personal_2019} states email addresses and phone numbers collected for safety or security purposes had been inadvertently used for advertising purposes.
This raises a number of issues, including privacy concerns, data protection concerns and ethical issues.
Also, this highlights Twitter's priority to provide data to advertising companies over protection of their users.

\section{Privacy Issues}
By allowing third-party advertisers to gain access to personal information, such as phone numbers and email addresses, user's privacy has been breached.
A user has the right to be control the manner in which their data is used, which is outlined in \textcite{twitter_privacy_2018}.
This right was not upheld by this breach, since the user submitted the data for a specific purpose, security or safety, but Twitter utilised the data for other purposes, advertising.

Data provided by a user for the purposes of security and safety must be treated as highly sensitive data, irrelevant of the content of the data.
In this case, the data was similar to data collected for adverting purposes, which could have possibly been the cause of the mishandling.
Twitter failed to protect user's highly sensitive data, and failed to inform directly impacted users.

\section{Data Protection Issues}
Twitter is a company whose user base spans the glob, therefore Twitter must abide by a range of data protection laws.
For example, for Twitter to operate within the UK, they must abide by \textcite{parliament_of_the_united_kingdom_data_2018} and \textcite{noauthor_regulation_2016}.
These laws are in place to protect data held by companies about individuals, which Twitter failed to uphold with allowing access to data to parties who should not have access.

One could deem Twitter's actions to be in breach of \textcite{noauthor_regulation_2016}.
\textcite{noauthor_regulation_2016} states personal data shall be processed in a transparent manner, meaning the user who's data is being process must be informed of how the data is being processed.
In this case, users were information the data in question was to be held for the sole use of user account security and safety.
Twitter made use of the data in question for advertising purposes, which was not divulged to users.
On the other hand, \textcite{support_personal_2019} claims allowing access to the data was inadvertent, which does not breach \textcite{noauthor_regulation_2016}.
Since this was not a breach of security, rather a internal mistake, it is not protected by \textcite{noauthor_regulation_2016}
Moreover, Twitter responsibly resolved the issue and informed its users in a timely manner \parencite{support_twitter_2019}, although Twitter has not directly informed users who's data was incorrectly handled.

\section{Ethical Issues}
Ethically, one could view the incorrect use of user's personal data as a terrible action by Twitter, since they have betrayed the trust of their users.
When data is given over to a company, the user providing the data is placing their trust in the company to make use of the data in the manner they describe.
Twitter failed to uphold this trust, and as such has failed their users.
Assuming \textcite{support_twitter_2019} is correct in stating this was unintentional, the loss of trust in Twitter should be minimal as the company is ran by humans, and the platform was built by humans.
Unlike machines, humans make mistakes, and users should trust in companies to correctly handle such mistakes by resolving the issue as soon as possible, and inform their users promptly.

On the other hand, one could view this as an ethically sound situation, because the data was given to Twitter, and Twitter should be allowed to make use of the data as they deem fit.
This would be a valid view to hold if \textcite{twitter_privacy_2018} did not outline the data Twitter can share with third-party companies.
The data outlined does not include email addresses or phone numbers, therefore such data should not have been disclosed to any third-parties.

One could see this situation as Twitter attempting to capitalise on extra data submitted by its user base in order to increase profits, and relations with advertising companies
\textcite{support_personal_2019} and \textcite{support_twitter_2019} both claim this is not the case, as it was inadvertent and unintentional.
Furthermore, such a business strategy is detrimental to the long term public view of Twitter, since such an act would eventually be made public, destroying their public image.
Moreover, this would be an extremely unethical method of increasing profits and relations with advertising companies, which could severely reduces relations with advertising companies, is such an act would be made public.

\section{Professional Issues}
From a professional stand point, Twitter correctly informed their users of the mishandled data, once the issue was resolved.
One could take the view Twitter waited too long between solving the issue and informing their users.
\textcite{support_twitter_2019} was posted three weeks after \textcite{support_personal_2019}.
If the issue was resolved, why did Twitter wait for three weeks before informing their users?
The large could be caused by Twitter informing the third-party advertising companies first, possibly to allow them time to remove the personal information from their own databases.
While there is no statement by Twitter on the cause of the lengthy delay, this is a likely reason as Twitter is a company with a single source of revenue, advertisements.

\section{Conclusion}

\printbibliography
\end{document}
