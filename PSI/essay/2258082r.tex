\documentclass[a4paper, 11pt]{article}
\usepackage[style = authoryear]{biblatex}
\usepackage[a4paper,includeheadfoot,margin = 2.54cm]{geometry}
\addbibresource{2258082r.bib}

\begin{document}
\title{IT professional issues raise by Twitter's unintentional incorrect use of personal information}
\date{\today}
\maketitle

\section{Introduction}
On the 8th of October 2019, Twitter delivered a public apology for unintentionally using user's personally information incorrectly \parencite{support_twitter_2019}.
\textcite{support_personal_2019} states email addresses and phone numbers collected for safety or security purposes had been inadvertently used for advertising purposes.
This raises a number of issues, including privacy concerns, data protection concerns and ethical issues.
Also, this highlights Twitter's priority to provide data to advertising companies over protection of their users.
Furthermore, Twitter has not received a fine for this, which raises the issue of the applicability of \textcite{_regulation_2016}.

\section{Privacy Issues}
By allowing third-party advertisers to gain access to personal information, such as phone numbers and email addresses, user's privacy has been breached.
A user has the right to control the manner in which their data is used, which is outlined in \textcite{twitter_privacy_2018}.
This right was not upheld by this breach, since the user submitted the data for a specific purpose, security or safety.
However, Twitter utilised the data for advertising, another purpose entirely.

Data provided by a user for the purposes of security and safety must be treated as highly sensitive data, irrelevant of the content of the data.
In this case, the data was similar to data collected for adverting purposes, which could have possibly been the cause of the mishandling.
Twitter failed to protect user's highly sensitive data and also did not inform users that were directly impacted.

\section{Data Protection Issues}
Twitter is a company whose user base spans the globe, therefore Twitter must abide by a range of data protection laws.
For example, for Twitter to operate within the UK, they must abide by \textcite{_data_2018} and \textcite{_regulation_2016}.
These laws are in place to protect individual's personal data, something Twitter failed to uphold by allowing outside parties access to data they should not have had access to.

One could deem Twitter's actions to be in breach of \textcite{_regulation_2016}.
GDPR states personal data shall be processed in a transparent manner, meaning the user who's data is being process must be informed of how the data is being processed.
In this case, users were information the data in question was to be held for the sole purpose of user account security and safety.
Twitter made use of the data in question for advertising purposes, which was not divulged to users.
On the other hand, \textcite{support_personal_2019} claims allowing access to the data was inadvertent, which does could be viewed as not breaching GDPR.
This would be a misunderstanding of GDPR.
The aim of GDPR is provide a unified set of data protection rules, and consequences for failing to abided by them.
As such even in the case of an inadvertent breach of privacy, GDPR still applies and such a breach would still warrant a fine of some description.
Furthermore, both security and privacy rules are outlined in GDPR, with Twitter explicitly breaking rules around purpose limitation.
Therefore, Twitter is deserving of a fine for breaching GDPR, even if no fine is currently issued.

\section{Ethical Issues}
Ethically, one could view the incorrect use of user's personal data as a terrible action by Twitter, since they have betrayed the trust of their users.
When data is given over to a company, the user providing the data is placing their trust in the company to make use of the data in the manner they describe.
Twitter failed to uphold this trust, and as such has failed their users.
Assuming \textcite{support_twitter_2019} is correct in stating this was unintentional, the loss of trust in Twitter should be minimal as the company is ran by humans, and the platform was built by humans.
Unlike machines, humans make mistakes.
Users should trust in companies to correctly handle such mistakes by resolving the issue as soon as possible, and inform their users promptly.

On the other hand, one could view this as an ethically sound situation.
The data was given to Twitter, and Twitter should be allowed to make use of the data as they deem fit.
This would be a valid view to hold if \textcite{twitter_privacy_2018} did not outline the data Twitter can share with third-party companies.
The data outlined does not include email addresses or phone numbers, therefore such data should not have been disclosed to any third-parties.

One could see this situation as Twitter attempting to capitalise on extra data submitted by its user base in order to increase profits, and relations with advertising companies.
\textcite{support_personal_2019} and \textcite{support_twitter_2019} both claim this is not the case, as it was inadvertent and unintentional.
Furthermore, such a business strategy is detrimental to the long term public view of Twitter, since such an act would eventually be made public, destroying their public image.
Moreover, this would be an extremely unethical method of increasing profits and relations with advertising companies.
In fact it could severely reduces relations with advertising companies if such an act would be made public.

\section{Professional Issues}
From a professional stand point, Twitter correctly informed their users of the mishandled data, once the issue was resolved.
One could take the view Twitter waited too long between solving the issue and informing their users.
\textcite{support_twitter_2019} was posted three weeks after \textcite{support_personal_2019} claims the issue was discovered.
The large delay could be caused by Twitter informing the third-party advertising companies first, possibly to allow them time to remove the personal information from their own databases.
While there is no statement by Twitter on the cause of the lengthy delay, this is a likely reason as Twitter is a company with a limited selection of revenue streams: advertising and data collection to name a few.

Being a large company which handles large amounts of user data, Twitter should have infrastructure in place to avoid a situation like this from occurring.
All data acquired should have been stored securely with access only allowed for services which require the data.
In this case, the data should have only been accessible from the service which provides account security.
Furthermore, all new services should have data audits.
These audits would ensure the data the new services claim to require is the correct and only data it will have access to.
For example, the advertising service should have been audited to ensure it only has access to data provided for advertising, not security purposes.

Twitter is not the only company to misuse user-provided security information for advertising purposes.
\textcite{venkatadri2019investigating} successfully proved Facebook uses phone numbers supplied for two factor authentication for advertising.
They state Facebook's privacy policy has very specific wording to allow for such uses, which could provide grounds for this to not be a breach of \textcite{_regulation_2016}.
\textcite{twitter_privacy_2018} is not worded to allow using security and safe information for advertising purposes, therefore this is a breach of GDPR.
This shows there is a possible issue with social media as a whole.
They exist solely to harvest vast amounts of user data, with little care given to the extent and breath data is collected.
Even to go as far as breaking laws and providing complex, maliciously crafted privacy policies, in order to maximise the amount of data collected.
The lack of a fine being issued to Twitter further incentives such acts.

\section{Conclusion}
The incorrect handling of user information intended for security and safety purposes by Twitter has not been in the company's best interest.
Failing to abided by data protection laws such as \textcite{_regulation_2016} could lead to large fines for the company.
Furthermore, Twitter should receive a fine for failing to abided by rules for purpose limitation.
Misusing user's security and safety information will cause users to lose trust in Twitter and its claims of transparency.
Damaging its user base directly and Twitter's credibility as a third-party authentication platform.
On a positive note, Twitter did resolve the issue and inform its user base of the issue, in a relatively timely manner \parencite{support_personal_2019}, \parencite{support_twitter_2019}.
Overall, user opinion of Twitter will have been damaged by this incident, Twitter has broken multiple privacy laws so should receive a fine, therefore Twitter should aim to avoid such incidents in the future.

\printbibliography
\end{document}
