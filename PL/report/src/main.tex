%! Author = stuart
%! Date = 18/03/2020

% Preamble
\documentclass[11pt]{article}

% Packages
\usepackage{amsmath}
\usepackage[a4paper,includeheadfoot,margin=2.54cm]{geometry}

\author{Stuart Reilly - 2258082}
\title{Programming Languages Assessed Coursework}
\date{\today}

% Document
\begin{document}
    \maketitle

    \section{Running the Code}\label{sec:running-the-code}
    Since you setup the project and all I had to do was extend it, you should know how to run the code.
    Furthermore, initial setup to get the code to compile was unnecessarily arduous due to the lackluster layout of the
    code, and the inclusion of a header statement in \texttt{Fun.g4}.

    \section{Syntactic Analysis}\label{sec:syntactic-analysis}
    To add the \texttt{for}-command, the grammer given in the assessed exercise was translate into ANTLR grammar, with
    \texttt{for} and \texttt{to} being added as lexer rules, rather than parser rules.

    The \texttt{switch}-command was added by first adding the overall structure of the grammar to the existing
    \texttt{com} rule, then adding parser rules for \texttt{case} and \texttt{default}, with \texttt{case} allowed
    zero or more times, and \texttt{default} being required.
    Since \texttt{case} statements could only allow literals and ranges, a new parser rule was added \texttt{raw\_lit}
    which duplicated the definition of \texttt{true}, \texttt{false} and \texttt{num} from \texttt{prim\_expr}, but would
    not allow non-literals by design.
    This moved checking for literal statements to the lexer.
    \texttt{switch}, \texttt{case} and \texttt{default} were also added as lexer rules.

    \section{Contextual Analysis}\label{sec:contextual-analysis}
    The \texttt{for}-command the initial value and terminal value to be of type \texttt{INT}, which was checked using
    existing \texttt{checkType} function.
    Furthermore, the iteration variable had to be already defined and of type \texttt{INT}, which was checked using the
    existing \texttt{retrieve} and \texttt{checkType} functions.
    Then the body is typed checked.

    The \texttt{switch}-command required the expression being tested to be either an \texttt{INT} or \texttt{BOOL}.
    This was tested with an extension to the \texttt{checkType} function, which allowed a range of possible types which
    the actual type could be.
    Since the lexer ensured the \texttt{case} guards could only be literals or ranges, there is no need to check that in
    the parser.
    The parser does check if the guards have the same type as the expression being tested by iterating over all
    \texttt{case}s and using the \texttt{checkType}.
    To check no guards overlap, a list of all guard values is collected, then each value is compared to each other
    value.
    The following cases are handled:
    \begin{itemize}
        \item both are literals: ensure they don't equal
        \item one is a literal and one is a range: ensure the literal doesn't fall within the range
        \item both are ranges: ensure the ranges don't overlap
    \end{itemize}
    Since the types are ensured to are checked to be the same prior to this iteration, handling \texttt{BOOL} and
    \texttt{INT} in the same way, by converting \texttt{BOOL} to integers for overlap checks.

    \section{Code Generation}\label{sec:code-generation}
    The \texttt{for}-command first evaluates the assignment's result, then stores that value into the iterator variable.
    The load and store opcodes are selected by determining if the address is global.
    If it is, use the \texttt{LOADG} and \texttt{STOREG} opcodes, otherwise use the \texttt{LOADL} and \texttt{STOREG}
    opcodes.
    Then the current offset is stored to be used for the end of block jump.
    A load is then outputted to load the iterator variable, followed by the evaluated limit expression.
    Then the iterator variable is compared to the limit expression.
    The current offset is remembered to update the conditional jump's target address with the address after the block.
    The conditional jump is then emitted, followed by the body of the loop.
    Next a load, increment and store are emitted to increment the iterator variable, followed by an unconditional jump
    to the remembered address before the iterator variable and evaluated limit expression.
    Finally, the conditional jumps's target address is updated to the current offset.

    The \texttt{switch}-command iterates over each case statement doing the following:
    \begin{itemize}
        \item Evaluate the switch expression
        \item If this is not the first case, update the previous case's fail jump
        \item Emit the case statement
        \item Check if \(1\) is at the top of the stack
        \item Remember the current offset as the fail jump offset
        \item Emit a conditional jump which jumps if \(1\) isn't at the top of the stack
        \item Remember the current offset as the success jump offset
        \item Emit an unconditional jump
    \end{itemize}
    If there was at least one case statement, update the last fail jump with the current offset, then emit the default
    case.
    Finally, update all case's success jumps target address with the current offset.

    If a case is a raw literal, the following occurs:
    \begin{itemize}
        \item Emit the literal value
        \item Compare if value equals the switch variable
        \item Remember the current offset as the success jump
        \item Emit a conditional jump which jumps if true
        \item Load the constant \(0\)
        \item Remember the current offet as the fail jump
        \item Emit an unconditional jump
        \item Update the success jump's target address as the current offset
        \item Evaluate the case block
        \item Load the constant \(1\)
        \item Update the fail jump's target address as the current offset
    \end{itemize}
    If a case is a range, the following occurs:
    \begin{itemize}
        \item Load the lower bound of the range
        \item Compare if the switch variable is equal to the lower bound
        \item Remember the current offset as the lower equality jump
        \item Emit a conditional jump which jumps if true
        \item Load the upper bound of the range
        \item Evaluate the switch variable
        \item Remember the current offset as the upper equality jump
        \item Emit a conditional jump which jumps if true
        \item
    \end{itemize}
    Each case finishes with either a \(0\) if it didn't match or \(1\) otherwise.
    This is checked before the next case statement to determine whether to try the next case or jump to the end of the
    switch, if the case did or didn't match respectively.
\end{document}