% Options for packages loaded elsewhere
\PassOptionsToPackage{unicode}{hyperref}
\PassOptionsToPackage{hyphens}{url}

\documentclass[]{article}

\usepackage{amssymb,amsmath}
\usepackage{booktabs}
\usepackage{etoolbox}
\usepackage{ifxetex,ifluatex}
\usepackage{lmodern}
\usepackage{tabulary}
\usepackage{xcolor}
\usepackage{listings}
\usepackage{listings-rust}

\ifnum 0\ifxetex 1\fi\ifluatex 1\fi=0 % if pdftex
  \usepackage[T1]{fontenc}
  \usepackage[utf8]{inputenc}
  \usepackage{textcomp} % provide euro and other symbols
\else % if luatex or xetex
  \usepackage{unicode-math}
  \defaultfontfeatures{Scale=MatchLowercase}
  \defaultfontfeatures[\rmfamily]{Ligatures=TeX,Scale=1}
\fi

% Use upquote if available, for straight quotes in verbatim environments
\IfFileExists{upquote.sty}{\usepackage{upquote}}{}
\IfFileExists{microtype.sty}{% use microtype if available
  \usepackage[]{microtype}
  \UseMicrotypeSet[protrusion]{basicmath} % disable protrusion for tt fonts
}{}

\makeatletter
\@ifundefined{KOMAClassName}{% if non-KOMA class
  \IfFileExists{parskip.sty}{%
    \usepackage{parskip}
  }{% else
    \setlength{\parindent}{0pt}
    \setlength{\parskip}{6pt plus 2pt minus 1pt}}
}{% if KOMA class
  \KOMAoptions{parskip=half}}
\makeatother

\IfFileExists{xurl.sty}{\usepackage{xurl}}{} % add URL line breaks if available
\IfFileExists{bookmark.sty}{\usepackage{bookmark}}{\usepackage{hyperref}}
\hypersetup{
  hidelinks,
  pdfcreator={LaTeX via pandoc}}
\urlstyle{same} % disable monospaced font for URLs

\lstset {
	language=Rust,
	tabsize=4,
	basicstyle=\footnotesize\ttfamily,
	keywordstyle=\bfseries\color{green!40!black},
	commentstyle=\itshape\color{purple!40!black},
	identifierstyle=\color{blue},
	stringstyle=\color{orange},
}

\author{2258082}
\date{\today}

\begin{document}

Please include your student number in filename (e.g. 1234567)

\begin{tabulary}{0.7\textwidth}{LL}
	\toprule
	Student number: & 2258082 \\
	\midrule
	Course title: & \\
	\midrule
	Questions answered: & \\
	\bottomrule
\end{tabulary}
\pagebreak

\section{Question 1}
\subsection{Question 1.a}
Rust aims to produce safe programs, which requires the complex type system.
Such a type system improves systems programming by ensuring only safe
operations can be performed, preventing undefined behaviour.
Since the majority of security vulnerabilities are due to unsafe memory
operations, preventing all unsafe operations will reduce the amount of security
vulnerabilities a program can have.
Traditional bugs such as use-after-free and dereferencing-null are impossible
in Rust.

Rust's type system and borrow checker result in a steep learning curve for new
comers to the language.
This is mediated by the book created by the Rust developers, the wide range of
guides for developers from other languages, and the descriptive error messages
from the compiler.
Furthermore, overcoming the learning curve teaches the programmer the range of
bugs that exist in other software that can be prevented by the Rust compiler,
leading to better software when the programmer uses other languages.

\subsection{Question 1.b}
One method is define each state of the state machine as its own struct, and
define the transitions between each state as functions.
Rust's ownership model allows the functions to take ownership of the state,
ensuring the state machine cannot exist in two states simultaneously.

The following code represents the state machine for a process.
\begin{lstlisting}
struct Building {...}
struct Ready {...}
struct Running {...}
struct Waiting {...}
struct Terminated {...}

impl Building {
	pub fn new() -> Building {...}
	pub fn build(self) -> Ready {...}
}

impl Ready {
	pub fn run(self) -> Running {...}
}

impl Running {
	pub fn interrupt(self) -> Ready {...}
	pub fn wait_for_event(self) -> Waiting {...}
	pub fn exit(self) -> Terminated {...}
}

impl Waiting {
	pub fn event_received(self) -> Ready {...}
}
\end{lstlisting}
Since each method representing a state transition consumes the current state
(takes \lstinline{self} rather than \lstinline{&self} or \lstinline{&mut self}),
the compiler can ensure the program cannot have a process in two states
simultaneously.
The compiler ensures each variable has a single owner, so the following code
would not compile.
\begin{lstlisting}
fn main() {
	let process = Building::new();
	let ready_process = process.build();
	let ready_process_again = process.build();
}
\end{lstlisting}
\lstinline{process.build()} takes ownership of the variable \lstinline{process},
so it cannot be used again in the function.

Another benefit to this method is it is impossible to enter a state incorrectly.
A state with a method which does not take \lstinline{self} as an argument is a
starting state, in this case \lstinline{Building} is the only start state.
No other states can be constructed directly, only through the defined state
transitions, and this can be enforced at compile time.
For example, it is impossible to create a process which is immediately in the
\lstinline{Running} state, without first being in the \lstinline{Building} and
\lstinline{Ready} states prior.

This method of representing a state machine relies on the compiler's ability to
track ownership of variables.
Without it, the programmer would be free to use a process in a state it is not
in.
Furthermore, checking if the process is in a valid state is a compile-time
check, which means there is no need to check which state a process is in at
runtime.
This is safer as runtime checks could be left out, and more performant as the
checks are done at compile time.
The main drawback with this method is the state transitions are not easy for
the programmer to follow.
State transitions are encoded as methods of a state, rather than an explicit
table.

\section{Question 2}
\subsection{Question 2.a}
The majority of an operating system can be implemented in a memory safe
language.
An area of difficulty is the boot section of the operating system.
When a computer boots, it's not in a state where high-level programming
languages can be ran.
For example, TLBs and caches may need to be initialised, or the processor may
not be in the correct mode.
These operations need to be done in assembly, which is inherently a memory
unsafe language.
Once the system has been initialised, operation can hand over to a memory safe
language for the majority of the execution.

\subsection{Question 2.b}
It would be possible but not a good idea.
While using a garbage collected language would allow the programmer to not
focus on memory as much, the drawbacks out way this.
Efficient garbage collectors require the memory to be split into 2 semispheres,
which would increase the memory usage of the operating system dramatically.
Furthermore, generally garbage collectors can interrupt the program at any time,
which is not ideal for an operating system.
The hardware would raise an interrupt while the garbage collector is running,
likely leading to an inconsistent state.
For example, interrupting the garbage collector to handle I/O, resulting in
more garbage being created which the garbage collector may not handle correctly
when control is returned to it.
Most garbage collectors are stop-the-world, meaning they pause the running
program and collect all garbage before unpauseing the program.
The time taken to collect the garbage is non-deterministic, leading to
arbitrary long pauses, which is not ideal for an operating system.
A possible solution is to use an incremental garbage collector, but this would
increase the number of context switches over time, leading to reduced overall
performance.

\subsection{Question 2.c}
There are some operations which require assembly code, therefore exclusively
writing all systems software in memory safe languages is unlikely.
Writing the majority of systems software in memory safe languages, and writing
safe wrappers to the sections which require unsafe languages is indeed
possible.

Migrating existing programs to a memory safe language requires large amounts of
work and training.
All developers on the project must be trained in the new language, and memory
safe languages like Rust have steep learning curves.
This will halt development of the program, so existing bugs will continue to
exist while the program is migrated.
Once the program has been migrated, the program is likely to be safer than
before.
There is no guarantee the migrations removes any bugs, and may even add new
bugs.
As such, industry would likely be reluctant to migrate existing programs.
This is compounded with the lack of clear financial benefit and clear financial
cost, caused by lost development on the product.

With that said, migrating to a memory safe language would reduce the likelihood
of certain bugs being introduced down the line.
Most memory safe languages have FFI (foreign function interface) support, which
can allow a program to either be incrementally migrated to a memory safe
language, or extended using a memory safe language.

\section{Question 3}
\subsection{Question 3.a}
Synchronous I/O blocks the calling thread while the operating system carries
out the I/O.
Asynchronous I/O does not block the calling thread, requiring the calling
thread to poll the operating system about the status of the I/O operation.

Using asynchronous I/O allows a program to continue running while the I/O is
being done.
This is beneficial in programs which interact with a user.
If the program uses synchronous I/O, the user interface would ``hang'' while the
program waited for the operating system to complete the I/O.
Using asynchronous I/O, the user interface would remain responsive during the
I/O.

\subsection{Question 3.b}
Coroutines allow the programmer to structure the program as if it was using
synchronous I/O, which is easy to reason about, and allow the runtime handle
the asynchronous nature of the underlying I/O operation.
The programmer must then ensure that each coroutine yields frequently to avoid
blocking the runtime.
Since only a single coroutine can run at a time on an OS thread, if one
coroutine doesn't yield for an extended period of time or blocks indefinitely,
the OS thread will block, until the coroutine yields or completes.
If the runtime only uses a single thread, the entire runtime will block.
This is known as cooperative multitasking, and has been avoided since a single
incorrect coroutine or program could block the whole program or system
respectively.
As such coroutines are a good method for small asynchronous programs, but large
programs would make better use of other techniques.
One technique is to dedicate a thread to polling the operating system for I/O
operations, all I/O operations are passed to that thread, returning a Future to
the calling thread.
The calling thread can then utilise existing methods of waiting for the Future
to complete, such as busy-waiting or semaphores.
This allows the operating system to schedule the threads, rather than a
programs runtime.

\subsection{Question 3.c}
Message passing does not prevent deadlocks.
If thread A required a message from thread B, but thread B required a message
for thread A before it could send a message to thread A, the system would
deadlock.
Depending on the implementation of the message passing, message passing does
not prevent data races.
If both actors are on the same system, often messages are passed using pointers.
In this case, if the sending actor modified the message while the receiving
actor was reading the message, a race condition can occur.

Designing a system to use lock-based synchronisation is complex because two
correctly locked block of code do not necessarily combine into a single
correctly locked block.
Whereas, a system designed with message-passing can compose assuming there are
no data races allowed by the implementation.
\end{document}
