\documentclass[11pt,a4paper,titlepage]{article}
\usepackage{graphicx}
\usepackage{tabulary}
\usepackage[english]{babel}
\usepackage{placeins}

\begin{document}
\title{Distributed and Parallel Technologies - Assessed Exercise Part 1}
\author{Stuart Reilly - 2258082R}
\date{\today}
\maketitle

\section{Comparative Sequential Performance}

\begin{table}[h]
	\centering
	\begin{tabular}{|l|l|l|}
		\hline
		Data Set & Go Runtimes & C Runtimes \\ \hline
		DS1 & 15.95 & 15.17s \\ \hline
		DS2 & 67.93 & 65.15s \\ \hline
		DS3 & 288.97 & 278.39 \\ \hline
	\end{tabular}
	\caption{Comparing C and Go Sequential Runtimes}
\end{table}

\section{Comparative Parallel Performance Measurements}
\subsection{Runtimes}

\includegraphics[width=\textwidth]{ds1}
\includegraphics[width=\textwidth]{ds2}
\includegraphics[width=\textwidth]{ds3}

\subsection{Speedups}

\includegraphics[width=\textwidth]{ds1-speed}
\includegraphics[width=\textwidth]{ds2-speed}
\includegraphics[width=\textwidth]{ds3-speed}

\subsection{Summary Data}

\begin{table}[!h]
	\centering
	\begin{tabulary}{\textwidth}{|L|L|L|L|L|}
		\hline
		Language & Sequential Runtime (s) & Best Parallel Runtime (s) & Best Speedup & No. Threads \\ \hline
		Go & 15.17 & 0.72 & 22.15 & 32 \\ \hline
		C + OpenMP & 15.95 & 0.64 & 23.70 & 32 \\ \hline
	\end{tabulary}
	\caption{Comparing C+OpenMP and Go Parallel Runtimes and Speedups for DS1}
\end{table}

\begin{table}[!h]
	\centering
	\begin{tabulary}{\textwidth}{|L|L|L|L|L|}
		\hline
		Language & Sequential Runtime (s) & Best Parallel Runtime (s) & Best Speedup & No. Threads \\ \hline
		Go & 67.93 & 3.10 & 21.91 & 48 \\ \hline
		C + OpenMP & 65.15 & 2.64 & 23.78 & 32 \\ \hline
	\end{tabulary}
	\caption{Comparing C+OpenMP and Go Parallel Runtimes and Speedups for DS2}
\end{table}

\begin{table}[!h]
	\centering
	\begin{tabulary}{\textwidth}{|L|L|L|L|L|}
		\hline
		Language & Sequential Runtime (s) & Best Parallel Runtime (s) & Best Speedup & No. Threads \\ \hline
		Go & 288.97 & 13.00 & 22.23 & 48 \\ \hline
		C + OpenMP & 278.39 & 11.52 & 24.17 & 32 \\ \hline
	\end{tabulary}
	\caption{Comparing C+OpenMP and Go Parallel Runtimes and Speedups for DS3}
\end{table}

\FloatBarrier
\subsection{Discussion}
C+OpenMP has a lower runtime than Go with every thread limit tested, and larger absolute speedup.
This is likely because C is a lower level language without a large runtime or garbage collector, and OpenMP can make better use of the available resources, when compared to Go.
With that said, the difference between C+OpenMP's and Go's absolute speedup is not as large as initially anticipated.
Go's implementation of coroutines (goroutines) and channels have been well designed and can make efficient use of the available resources.
The garbage collector in Go reserves CPU cores for its concurrent mark-and-sweep phases, which will negatively impact the program's multithreaded performance, as less CPU cores will be available for the program's goroutines.

\section{Programming Model Comparison}
C+OpenMP is a higher level model for concurrent programming than Go's goroutinues.
This has advantages, such as better guarantees for resource utilisation, lower barrier to entry and lower cognitive load in order to use.
Furthermore, C+OpenMP allows sequential code to be parallelised with minimal changes to the existing code, where as Go requires all methods to be modified to pass return variables into a channel.

During testing, Go spent more time in kernel space than C+OpenMP.
This is likely because, C+OpenMP creates its thread pool once at the first parallel pragma, whereas Go builds its thread pool as more goroutines are created.
Also, the Go garbage collector could be making syscalls in order to free unused variables during either its 'stop the world' phase or 'sweep' phase.

\end{document}
