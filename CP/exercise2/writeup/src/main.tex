%! Author = stuart
%! Date = 09/03/20

% Preamble
\documentclass[11pt]{article}

% Packages
\usepackage{amsmath}
\title{Constraint Programming Exercise 2\\The Balanced Academic Curriculum Problem}
\date{\today}
\author{Stuart Reilly - 2258082R}

% Document
\begin{document}
    \maketitle

    \section{Model}\label{sec:model}
    The model is defined as the two-dimensional array \texttt{table} of \texttt{IntVar} where each \texttt{IntVar} has
    the domain of \({0,1}\), the \texttt{i}\textsuperscript{th} \texttt{j}\textsuperscript{th} \texttt{IntVar} has the
    value of \(1\) if the \texttt{j}\textsuperscript{th} course is in the \texttt{i}\textsuperscript{th} period.
    From this, the \texttt{IntVar[]} \texttt{creditsInPeriod} is defined as the i\textsuperscript{th} element in
    \texttt{creditsInPeriod} is sum of the credits for all courses in the \texttt{i}\textsuperscript{th} period.
    The \texttt{IntVar}s \texttt{maxCreditsInAllPeriods} and \texttt{minCreditsInAllPeriods} are then defined as the
    maximum and minimum in \texttt{creditsInPeriod}.
    From these two variables, the \texttt{IntVar} \texttt{imbalance} is defined to be the difference between
    \texttt{maxCreditsInAllPeriods} and\\\texttt{minCreditsInAllPeriods}.

    There is a constraint which ensures a course can only appear in a single period is the sum of each column in
    \texttt{table} must equal \(1\).
    To ensure the course count requirements for each period are met, the sum of each row must be within the external
    variables \texttt{minCourses} and \texttt{maxCourses}.
    Similarly, credit count requirements are enforced by requiring the sum of each row's credits to be with the external
    variables \texttt{minCredits} and \texttt{maxCredits}.
    To enforce the prerequisites, if course \texttt{i} has a prerequisite \texttt{j} then the \texttt{i}
    \textsuperscript{th} column of \texttt{table} must be lexographically less then the \texttt{j}
    \textsuperscript{th} column of t\texttt{table}.
    Since the columns can be represented as a binary number, if one column is lexicographically less than another, the
    other column can be a perquisite of the first.

    \section{Computational Experience}\label{sec:computational-experience}
    Through trial and error, the default search strategy produced the least nodes.
    Since the model is defined as an \(MxN\) array, the memory use of the model scales with the product of the number of
    courses and periods.
    The computational complexity growth for ensuring prerequisites are met grows linearly with the number of
    prerequisites.

    \section{Alternative Model}\label{sec:alternative-model}
    Rather than using a two-dimensional array, an array of sets could be used.
    The \texttt{i}\textsuperscript{th} set in the array would be the set of courses in the \texttt{i}
    \textsuperscript{th} period.
    This method would require a more complex method of checking prerequisites as for each course with prerequisites,
    each prerequisite would have to be searched for in all prior periods, rather than comparing two binary values.
\end{document}